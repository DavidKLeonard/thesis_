% Chapter 1

\chapter{Target and Motivation} % Chapter title

\label{ch:target} % For referencing the chapter elsewhere, use \autoref{ch:introduction} 

%----------------------------------------------------------------------------------------

\noindent The cleavage of C(sp$^3$)$-$C(sp$^3$) is arguably one of the most challenging transformations in chemistry and doing so is highly difficult to achieve in a mild and selective manner that is compatible with organic synthetic strategies. C$-$C single bond cleavage reactions are of the utmost importance in the petrochemical industry where crude oil is \textit{cracked} into an assortment of smaller hydrocarbon fractions for further processing and for maximising of value. Whilst these reactions are carried out routinely in the industrial sector using heterogeneous catalysts, the low selectivity and harsh reaction conditions necessitated ($>300$\textsuperscript{$\circ$}C) renders this an unsuitable solution for the synthesis of a diverse range of functionalised organic compounds bearing sensitive chemical motifs.

Homogeneous catalysts generally offer the attractive properties of being both highly active and highly selective (cf. heterogeneous catalysts) whilst allowing more potential for catalyst \textit{tuning} and greater mechanistic insight. In contrast to heterogeneous systems, where reactions take place on non-ideal catalyst surfaces, homogeneous systems typically exist in solution as a single well-defined molecular species with only one available reaction site available, resulting in fewer undesired byproducts.

Currently there are very few literature examples of selective cleavage of C(sp\textsuperscript{3})--C(sp\textsuperscript{3}) bonds under mild reaction conditions. Success in this area may provide the potential for: (i) the synthesis of drug metabolites to enable more rapid selection of viable clinical candidates; (ii) improved energy efficiency and selectivity of fossil fuel cracking; (iii) providing renewable chemical feedstocks via the valorisation of biomass. In this thesis, different protocols for the activation of C(sp$^3$)$-$C(sp$^3$) bonds are presented. In each case, $3$d metals are used as catalysts facilitating the oxidation processes, and molecular oxygen---air, in fact---is used as the sole oxidant.

\sloppy{
Aside from that which is outlined above, other topics evolved which I contributed to. Also included in this thesis, therefore, is the development of a nickel-based heterogeneous catalyst for reductive dehalogenation of aryl halides.
}
