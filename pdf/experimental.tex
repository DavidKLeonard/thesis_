\subsection{General Procedure for the Synthesis of Amines (GP-B)}

\label{ss:GPB}

To an 8 mL glass vial equipped with a magnetic stir bar, Pd/C (Palladium on activated charcoal 10 \% Pd basis (from Sigma-$-$Aldrich)) (40 mg), and ketone (2.0 mmol) were added. The vial was capped and pierced with a small needle before EtOH (3 mL) was added. The vial was then placed into an aluminium heating block and then sealed inside a 300 mL steel autoclave (Parr Instrument Company). The autoclave was flushed with H$_2$ three times and then pressurised to the desired value (50 bar). Then it was placed into an aluminium block and heated to the to 130$^{\circ}$C. At the end of the reaction, the autoclave was quickly cooled to room temperature. A sample of the the reaction mixture was analysed by GC-FID . The product was purified via flash column chromatography using heptane/ethyl acetate. Solvent was removed \textit{in vacuo} to yield the desired product.

\subsection{General Procedure for the Synthesis of Amines (GP-C)}

\label{ss:GPC}

To a 100 mL round bottom flask equipped with magnetic stir bar was added amine (8.0 mmol, 1 equiv) and methanol (20 mL). Next, aldehyde (10.0 mmol, 1.25 equiv) was added, followed by NaBH$_3$CN (10.0 mmol, 1.25 equiv). After completion, the reaction was quenched with saturated NaHCO$_3$ (20 mL) and methanol was removed \textit{in vacuo}. The mixture was diluted with H$_2$O and extracted with ethyl acetate (50 mL $\times$3). The extracts were combined and washed with brine and dried over anhydrous Na$_2$SO$_4$ and filtered. Lastly, the solvent was removed \textit{in vacuo}.

\subsection{General Procedure for the Synthesis of Amines (GP-D)}

\label{ss:GPD}

To a 100 mL round bottom flask equipped with magnetic stir bar was added dimethylformamide (20 mL), bromide (20.0 mmol, 1 equiv) and amine (30.0 mmol, 2.0 equiv). The reaction mixture was heated to 100$^{\circ}$C. After completion the reaction mixture was filtered to remove K$_2$CO$_3$. The mixture was diluted with H$_2$O and extracted with ethyl acetate (50 mL $\times$ 3). The extracts were combined and washed
with brine and dried over anhydrous Na$_2$SO$_4$ and then filtered. Lastly, solvent and amine was removed \textit{in vacuo}.

\subsection{General Procedures for Copper-Catalyzed Oxidations}

\subsubsection{General Procedure E (GP-E)}

\label{ss:GPE}

To a 4 mL vial equipped with a magnetic stir bar, CuCl (2.5 mg, 0.025 mmol) was added. The vial was capped and pierced with a small needle. HPLC grade acetonitrile (2 mL), pyridine (80 $\mu$g) and tertiary amine, 0.5 mmol were added, independently. The vial was then placed into an aluminium heating block and then sealed inside a 300 mL steel autoclave (Parr Instrument Company). The autoclave was flushed with air twice and then pressurised to the desired value (30 bar). Then it was placed into an aluminium block and heated to the to 100$^{\circ}$C. At the end of the reaction, the autoclave was quickly cooled to room temperature. A sample of the the reaction mixture was analysed by GC-FID . The product was purified via flash column chromatography using heptane/ethyl acetate. Solvent was removed \textit{in vacuo} to yield the desired product.

\subsubsection{General Procedure F (GP-F)}

\label{ss:GPF}

To a 4 mL vial equipped with a magnetic stir bar, Cu(OTf)$_2$ (6.0 mg) was added. The vial was capped and pierced with a small needle. HPLC grade acetonitrile (2 mL), pyridine (80 $\mu$L) and \textit{N}-phenylmorpholine, 0.5 mmol were added, independently. The vial was then placed into an aluminium heating block and then sealed inside a 300 mL steel autoclave (Parr Instrument Company). The autoclave was flushed with air twice and then pressurised to the desired value (20 bar). Then it was placed into an aluminium block and heated to the to 80$^{\circ}$C. At the end of the reaction, the autoclave was quickly cooled to room temperature. A sample of the the reaction mixture was analysed by GC-FID . The product was purified via flash column chromatography using heptane/ethyl acetate. Solvent was removed \textit{in vacuo} to yield the desired product.

