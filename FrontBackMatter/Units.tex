% Summary

\pdfbookmark[1]{Units of Measurement}{Units of Measurement} % Bookmark name visible in a PDF viewer

%\begin{flushright}{\slshape    
%We have seen that computer programming is an art, \\ 
%because it applies accumulated knowledge to the world, \\ 
%because it requires skill and ingenuity, and especially \\
%because it produces objects of beauty.} \\ \medskip
%--- \defcitealias{knuth:1974}{Donald E. Knuth}\citetalias{knuth:1974} \citep{knuth:1974}
%\end{flushright}

%\bigskip



%\bigskip

%----------------------------------------------------------------------------------------

\begingroup

\let\clearpage\relax
\let\cleardoublepage\relax
\let\cleardoublepage\relax

\chapter*{Units of Measurement}


\noindent The International System of Units (SI) is utilised throughout this work to measure experimental of theoretical quantities. All derived units and their expression in terms of the SI bas units are given below.\bigskip

\begin{table}[h!]
\centering
\begin{tabular}{ l l l l }
\toprule
quantity & unit & name & conversion to SI base units \\
\midrule
temperature & $^{\circ}$C & degrees celsius &T/K$=$T/$^{\circ}$C$-273.15$ \\
volume & mL & millilitre & $1$ mL = $1$ cm$^3$ = $10^{-6}$ m$^3$ \\
\bottomrule
\end{tabular}
\label{tab:SI}
\end{table}

%\noindent\emph{Regarding \mLyX}: The \mLyX\ port was initially done by
%\emph{Nicholas Mariette} in March 2009 and continued by
%\emph{Ivo Pletikosi\'c} in 2011. Thank you very much for your work and the contributions to the original style.

\endgroup
